\documentclass{article}%
\usepackage[T1]{fontenc}%
\usepackage[utf8]{inputenc}%
\usepackage{lmodern}%
\usepackage{textcomp}%
\usepackage{lastpage}%
%
\title{Speech Deterioration in COPD}%
\author{Piyush Gupta}%
\date{}%
%
\begin{document}%
\normalsize%
\maketitle%
\section{Abstract}%
\label{sec:Abstract}%
\begin{abstract}
    
    
    Transmit antenna selection (TAS) is a technique that achieves better performance than a single
    antenna system while using the same number of radio frequency chains. We propose a novel TAS
    rule called the lam-weighted interference indicator rule (LWIIR). We prove that for the general class of fading models with continuous cumulative distribution functions, LWIIR achieves the lowest average symbol error probability (SEP) among all TAS rules for  an underlay cognitive radio system that employs binary power control and is subject to the interference-outage constraint. This constraint imposes a limit on the probability that the interference power at the primary exceeds a threshold. It is a generalization of the widely studied peak interference constraint. We derive the average SEP of LWIIR. The insightful performance analysis applies to any number of transmit and receive antennas and to many constellations.  We also analyze the practical scenario in which the secondary transmitter has imperfect information of the channel gains from itself to the secondary and primary receivers. We show that the imperfections in these two sets of channel gains have different impacts on the system. Our benchmarking shows that LWIIR outperforms many selection rules considered in the literature.
    SSD
        Principle of mes (POM)
\end{abstract}

%
Research suggests that speech deterioration indicates an exacerbation in patients with chronic obstructive pulmonary disease (COPD). This study provides a comparison of read speech of 9 stable COPD patients and 5 healthy controls (I) and 9 stable COPD patients and 9 COPD patients in exacerbation (II). Results showed a significant effect of condition on the number of (non{-}linguistic) in{-} and exhalations per syllable (I, II) and the ratio of voiced and silence intervals (II). Also, sustained vowels by 10 COPD patients in exacerbation were compared with 10 vowels in stable condition (III). Results showed an effect of condition on duration, shimmer, harmonics{-}to{-}noise ratio (HNR), and voice breaks. It was concluded that HNR, vowel duration, and the number of (non{-}linguistic) in{-} and exhalations per syllable show potential for remote monitoring. Further research is needed to examine the validity of the results for natural speech and larger sample sizes.%
\section{Introduction}%
\label{sec:Introduction}%

%
This is the introduction section. Add your content here.%
\section{Methodology}%
\label{sec:Methodology}%

%
Describe your methodology in this section.%
\section{Results}%
\label{sec:Results}%

%
Present your results in this section.%
\section{Conclusion}%
\label{sec:Conclusion}%

%
Summarize your findings and conclude the study.%
\end{document}